\documentclass[fontsize=12pt,paper=a4,twoside=semi,parskip=half-,headsepline,headinclude]{scrreprt}
% Grundgröße 12pt, zweiseitig
\usepackage[headsepline,automark]{scrlayer-scrpage}
% Seitenköpfe automatisch 
\usepackage[ngerman]{babel}
% Sprachpaket für Deutsch (Umlaute, Trennung,deutsche Überschriften)
\usepackage{blindtext}
% macht nur den Blindtext, den Sie aktuell sehen
\usepackage{lmodern}
% schöne PDF-Schrift
\usepackage{graphicx,hyperref,amssymb}
%Graphikeinbindung, Hyperref (alles klickbar, Bookmarks),
%Math. Symbole aus AmsTeX
\usepackage[utf8]{inputenc}
% Umlaute und über Tastatur einzugeben
\usepackage{listings}
% nette Listing-Formatierung
\usepackage[backend=biber]{biblatex}
\addbibresource{literatur.bib}

% Festlegung Kopf- und Fußzeile     
\defpagestyle{meinstil}{%
{\headmark \hfill}
{\hfill \headmark}
{\hfill \headmark\hfill }
(\textwidth,.4pt)
}{%
(\textwidth,.4pt)
{\pagemark\hfill Vorname Name}
{\today \hfill \pagemark}
{\today\hfill\pagemark} 
}
\pagestyle{meinstil} 

\raggedbottom
\renewcommand{\topfraction}{1}
\renewcommand{\bottomfraction}{1}

 
%%%%%%%%%%%%%%%%%%%%%%%%%%%%%%%%%%%%%%%%%%%%%%%%%%%%%%%%%%%%%%%%%%%%%%%%%%
\begin{document}    % hier gehts los
  \thispagestyle{empty} % Titelseite
\includegraphics[width=0.2\textwidth]{hsh_icons/Wortmarke_WI_schwarz}

   {  ~ \sffamily
  \vfill
  {\Huge\bfseries Performance-Vergleich von \\Nebenläufigkeits-Konzepten in \\objektorientierten Programmiersprachen}
  \bigskip

  {\Large 
  Philipp Ermer \\[2ex]
 Master-Arbeit im Studiengang "`Angewandte Informatik"' 
 \\[5ex]
   \today } 
}
 \vfill
  
  ~ \hfill
  \includegraphics[height=0.3\paperheight]{hsh_icons/H_WI_Pantone1665} 

\vspace*{-3cm}


\newpage 
\thispagestyle{empty}
\quad 


  \newpage \thispagestyle{empty}
 \begin{tabular}{ll}
{\bfseries\sffamily Autor} &  Philipp Ermer \\ 
            & Matrikelnummer: 1313395 \\
            & philipp.ermer@web.de \\[5ex]
{\bfseries\sffamily Erstprüferin:} & Prof. Dr. Holger Peine \\
          & Abteilung Informatik, Fakultät IV \\
         & Hochschule Hannover \\
        & holger.peine@hs-hannover.de \\[5ex]
{\bfseries\sffamily Zweitprüfer:} &Prof. Dr. Robert Garmann \\
          & Abteilung Informatik, Fakultät IV \\
         & Hochschule Hannover \\
        & robert.garmann@hs-hannover.de
\end{tabular}

\vfill

%%%%%%%%%%%%%%%%%%%%%%%%%%%%%%%%%%%%%%%%%%%%%%%%%%%%%%%%%%%%%%%%%%%%%%%%%%%%%%%
% Die folgende Lizenz kann die weitere Verwertung Ihrer Arbeit vereinfachen.
% Wenn Sie Ihre Arbeit nicht unter dem genannten Lizenzvertrag lizenzieren
% möchten, können Sie diesen Abschnitt entfernen.
% Dies hat keinerlei Einfluss auf die Bewertung Ihrer Arbeit.
Soweit nicht anders gekennzeichnet, ist dieses Werk unter einem
Creative-Commons-Lizenzvertrag Namensnennung 4.0 lizenziert.
Dies gilt nicht für Zitate und Werke, die aufgrund einer anderen Erlaubnis
genutzt werden.
Um die Bedingungen der Lizenz einzusehen, folgen Sie bitte dem Hyperlink:\\
\url{https://creativecommons.org/licenses/by/4.0/deed.de}

\vfill
%%%%%%%%%%%%%%%%%%%%%%%%%%%%%%%%%%%%%%%%%%%%%%%%%%%%%%%%%%%%%%%%%%%%%%%%%%%%%%%

\begin{center} \sffamily\bfseries Selbständigkeitserklärung \end{center}
% fett und zentriert in der minipage

Hiermit erkläre ich, dass ich die eingereichte Master-Arbeit
selbständig und ohne fremde Hilfe verfasst, andere als die von mir angegebenen Quellen
und Hilfsmittel nicht benutzt und die den benutzten Werken wörtlich oder
inhaltlich entnommenen Stellen als solche kenntlich gemacht habe. 
\vspace*{7ex}

Hannover, den \today \hfill Unterschrift


\newpage 
\thispagestyle{empty}
\quad 
\newpage


  \pdfbookmark[0]{Inhalt}{contents}
  \tableofcontents  % Inhaltsverzeichnis

\listoffigures      % Abbildungsverzeichnis

\listoftables       % Tabellenverzeichnis

\chapter{Einführung}

\section{Motivation}

\section{Ziel der Arbeit}

\section{Verwandte Arbeiten}

\section{Methodik}

\section{Gliederung und Aufbau}



\chapter{Nebenläufigkeitskonzepte}

TODO: Einleitungs-Text Kapitel Nebenläufigkeit

\section{Java: Plattform Threads}

abc

\section{Java: Virtual Threads}

Java Virtual Threads sind seit dem JDK 21(Java Development Kit) [Quelle JDK 21] ein fester Bestandteil von Java. Ursprünglich entwickelt wurden die Virtual Threads im Rahmen des Project Loom als so genannte Fiebers. Das Ziel war es die mittlerweile in die Jahre gekommenen Java Threads, durch eine leichtgewichtigere Variante zu ersetzen, welche auf die gleiche API(Application Programming Interface) zurückgreift.

Die bisher verwendeten Java Threads werden auch als Platform Threads bezeichnet, da sie sogenannte Wrapper sind die einen Betriebssystem Thread umschließen. Das bedeutet, dass für jeden Platform Thread ein Betriebssystem Thread erstellt wird, welcher dann von einem Java Platform Thread umschlossen wird. Das hat zur Folge, dass Plattform Threads schwergewichtig werden, da Betriebssystem Thread ressourcenintensiv sind und ihre maximale Anzahl damit begrenzt ist. Dies führt dazu, dass die Anzahl der Threads, weit vor anderen Ressourcen, zum limitierenden Faktor wird. Ein weiteres Problem ist, dass die Verwaltung der Threads
vom Betriebssystem übernommen wird und die JRE(Java Runtime Enviroment) keinen Einfluss darauf nehmen kann. Die Verwaltung der Threads auf Betriebssystemebene muss jedoch sehr allgemein gehalten werden, da sie nicht nur für Java, sonder für alle Programmiersprachen funktionieren muss.

TODO: Pooling? Senkt kosten zum erstellen, erhört nicht die maximale Anzahl an Threads

TODO: Zeichnung Platform Threads

Ziel von Virtual Threads ist es dies Schwachstellen zu umgehen, in dem eine große Zahl Virtual Threads auf eine kleine Zahl Betriebssystem Threads verteilt wird. Die Verwaltung der Virtual Threads kann so mit von der JRE übernommen werden.

Im Detail sieht dies wie folgt aus:

Es wird eine kleine Anzahl an Platform Threads erstellt, die ihrer seit einen Betriebssystem Thread umschließen(Wrapper). Diese Platform Threads werden als Carrier bezeichnet und ihre Anzahl entspricht standardmäßig der Anzahl der Prozessorkerne des ausführenden Systems. Die Carrier Threads werden von einem ForkJoinPool verwalten der nach dem FIFO(First In - First Out) Prinzip arbeitet. 

carrier thread
forkjoin pool
Workstealing

VT mount/ unmount
JDK rewritten where blocking
continuations .yield()
blocking I/O operation automatically suspends the virtual thread until it can be resumed later

pinning (schlecht)
VT kann nicht unmounted werden
native call
synchronized code block
-kurze Zeit kein Problem
-sonst reentrant lock
-zukünftig vermutlich gelöst

Thread Per Request Model
Gut für nebenläufige IO Operationen
Request=PT=OST schlecht
Request=VT gut
task ganze zeit im gleiche VT
wird in unterschiedliche PT mounted
the virtual thread consumes an OS thread only while it performs calculations on the CPU

Little's law
latency, concurrency, and throughput

platform threads, implemented as wrappers for OS threads (1:1 scheduling). Virtual threads employ M:N scheduling, where a large number (M) of virtual threads is scheduled to run on a smaller number (N) of OS threads.

VT seit JDK 19 previe(alte JEP)
Einsatz auf mehrkernprozessoren

Einfache Programmierung
Gleiche Trhread API
java.lang.Thread API
Können integriert werden ohne bestehende Codebasis wesentlich zu verändern
TODO:Code Beispiel
Virtual threads are cheap and plentiful, and thus should never be pooled: 
A new virtual thread should be created for every application task
virtual threads will thus be short-lived and have shallow call stacks
Platform threads, by contrast, are heavyweight and expensive


PT erstellen kostet Zeit Speicher
2 MB pro PT
~ 1 Ms
~100 ns context switch(OS abhängig)

PT
einige hunder
VT
Millionen(fast unbegrenzt)

Virtual Threads \cite{Karsten2020}

\section{Java: Completable Futures}

\includegraphics[scale=2.5]{figures/example.png}

\section{Kotlin: CO-Routinen}

\section{Kotlin: Threads}

\section{Go: Go-Routinen}

\section{C\#: async/await}

\section{Vergleich der Konzepte}



\chapter{Benchmarks}

\section{Benchmark 1}

\section{Benchmark 2}

\section{Benchmark 3}

\section{Benchmark 4}



\chapter{Testaufbau}

\section{Testumgebung}

\section{Performancefaktoren}

\section{Werkzeuge zur Datenerhebung}



\chapter{Messungen}

\section{Messung: Benchmark 1}

\section{Messung: Benchmark 2}

\section{Messung: Benchmark 3}

\section{Messung: Benchmark 4}



\chapter{Auswertung}

\section{Ergebnisse und Beobachtungen}

\section{Diskussion und Bewertung}



\chapter{Zusammenfassung und Ausblick}

\section{Zusammenfassung}

\section{Ausblick}



\printbibliography


%\input{abkuerz.tex}      % Einbinden von Tex-Files
%\input{einfuehrung.tex}
%
%\include{normen}        % Einbinden von größeren Tex-Files,z.B. Kapiteln
%\include{aufbau}
%\include{zitieren}
%\include{form}
%\include{allgtips}
%
%\bibliographystyle{alpha}  % Schlüssel als Buchstaben
%\bibliography{literaturverzeichnis}      % Literaturverzeichnis

\end{document}


