\documentclass[fontsize=12pt,paper=a4,twoside=semi,parskip=half-,headsepline,headinclude]{scrreprt}
% Grundgröße 12pt, zweiseitig
\usepackage[headsepline,automark]{scrlayer-scrpage}
% Seitenköpfe automatisch 
\usepackage[ngerman]{babel}
% Sprachpaket für Deutsch (Umlaute, Trennung,deutsche Überschriften)
\usepackage{blindtext}
% macht nur den Blindtext, den Sie aktuell sehen
\usepackage{lmodern}
% schöne PDF-Schrift
\usepackage{graphicx,hyperref,amssymb}
%Graphikeinbindung, Hyperref (alles klickbar, Bookmarks),
%Math. Symbole aus AmsTeX
\usepackage[utf8]{inputenc}
% Umlaute und über Tastatur einzugeben
\usepackage{listings}
% nette Listing-Formatierung
\usepackage[backend=biber]{biblatex}
\addbibresource{literatur.bib}

% Festlegung Kopf- und Fußzeile     
\defpagestyle{meinstil}{%
{\headmark \hfill}
{\hfill \headmark}
{\hfill \headmark\hfill }
(\textwidth,.4pt)
}{%
(\textwidth,.4pt)
{\pagemark\hfill Vorname Name}
{\today \hfill \pagemark}
{\today\hfill\pagemark} 
}
\pagestyle{meinstil} 

\raggedbottom
\renewcommand{\topfraction}{1}
\renewcommand{\bottomfraction}{1}

 
%%%%%%%%%%%%%%%%%%%%%%%%%%%%%%%%%%%%%%%%%%%%%%%%%%%%%%%%%%%%%%%%%%%%%%%%%%
\begin{document}    % hier gehts los
  \thispagestyle{empty} % Titelseite
\includegraphics[width=0.2\textwidth]{hsh_icons/Wortmarke_WI_schwarz}

   {  ~ \sffamily
  \vfill
  {\Huge\bfseries Performance-Vergleich von \\Nebenläufigkeits-Konzepten in \\objektorientierten Programmiersprachen}
  \bigskip

  {\Large 
  Philipp Ermer \\[2ex]
 Master-Arbeit im Studiengang "`Angewandte Informatik"' 
 \\[5ex]
   \today } 
}
 \vfill
  
  ~ \hfill
  \includegraphics[height=0.3\paperheight]{hsh_icons/H_WI_Pantone1665} 

\vspace*{-3cm}


\newpage 
\thispagestyle{empty}
\quad 


  \newpage \thispagestyle{empty}
 \begin{tabular}{ll}
{\bfseries\sffamily Autor} &  Philipp Ermer \\ 
            & Matrikelnummer: 1313395 \\
            & philipp.ermer@web.de \\[5ex]
{\bfseries\sffamily Erstprüferin:} & Prof. Dr. Holger Peine \\
          & Abteilung Informatik, Fakultät IV \\
         & Hochschule Hannover \\
        & holger.peine@hs-hannover.de \\[5ex]
{\bfseries\sffamily Zweitprüfer:} &Prof. Dr. Robert Garmann \\
          & Abteilung Informatik, Fakultät IV \\
         & Hochschule Hannover \\
        & robert.garmann@hs-hannover.de
\end{tabular}

\vfill

%%%%%%%%%%%%%%%%%%%%%%%%%%%%%%%%%%%%%%%%%%%%%%%%%%%%%%%%%%%%%%%%%%%%%%%%%%%%%%%
% Die folgende Lizenz kann die weitere Verwertung Ihrer Arbeit vereinfachen.
% Wenn Sie Ihre Arbeit nicht unter dem genannten Lizenzvertrag lizenzieren
% möchten, können Sie diesen Abschnitt entfernen.
% Dies hat keinerlei Einfluss auf die Bewertung Ihrer Arbeit.
Soweit nicht anders gekennzeichnet, ist dieses Werk unter einem
Creative-Commons-Lizenzvertrag Namensnennung 4.0 lizenziert.
Dies gilt nicht für Zitate und Werke, die aufgrund einer anderen Erlaubnis
genutzt werden.
Um die Bedingungen der Lizenz einzusehen, folgen Sie bitte dem Hyperlink:\\
\url{https://creativecommons.org/licenses/by/4.0/deed.de}

\vfill
%%%%%%%%%%%%%%%%%%%%%%%%%%%%%%%%%%%%%%%%%%%%%%%%%%%%%%%%%%%%%%%%%%%%%%%%%%%%%%%

\begin{center} \sffamily\bfseries Selbständigkeitserklärung \end{center}
% fett und zentriert in der minipage

Hiermit erkläre ich, dass ich die eingereichte Master-Arbeit
selbständig und ohne fremde Hilfe verfasst, andere als die von mir angegebenen Quellen
und Hilfsmittel nicht benutzt und die den benutzten Werken wörtlich oder
inhaltlich entnommenen Stellen als solche kenntlich gemacht habe. 
\vspace*{7ex}

Hannover, den \today \hfill Unterschrift


\newpage 
\thispagestyle{empty}
\quad 
\newpage


  \pdfbookmark[0]{Inhalt}{contents}
  \tableofcontents  % Inhaltsverzeichnis

\listoffigures      % Abbildungsverzeichnis

\listoftables       % Tabellenverzeichnis

\chapter{Einführung}

\section{Motivation}

\section{Ziel der Arbeit}

\section{Verwandte Arbeiten}

\section{Methodik}

\section{Gliederung und Aufbau}



\chapter{Nebenläufigkeitskonzepte}

\section{Java: Virtual Threads}

Virtual Threads \cite{Karsten2020}

\section{Java: Plattform Threads}

\includegraphics[scale=2.5]{figures/example.png}

\section{Java: Completable Futures}

Test 2

\section{Kotlin: CO-Routinen}

\section{Kotlin: Threads}

\section{Go: Go-Routinen}

\section{C\#: async/await}

\section{Vergleich der Konzepte}



\chapter{Benchmarks}

\section{Benchmark 1}

\section{Benchmark 2}

\section{Benchmark 3}

\section{Benchmark 4}



\chapter{Testaufbau}

\section{Testumgebung}

\section{Performancefaktoren}

\section{Werkzeuge zur Datenerhebung}



\chapter{Messungen}

\section{Messung: Benchmark 1}

\section{Messung: Benchmark 2}

\section{Messung: Benchmark 3}

\section{Messung: Benchmark 4}



\chapter{Auswertung}

\section{Ergebnisse und Beobachtungen}

\section{Diskussion und Bewertung}



\chapter{Zusammenfassung und Ausblick}

\section{Zusammenfassung}

\section{Ausblick}



\printbibliography


%\input{abkuerz.tex}      % Einbinden von Tex-Files
%\input{einfuehrung.tex}
%
%\include{normen}        % Einbinden von größeren Tex-Files,z.B. Kapiteln
%\include{aufbau}
%\include{zitieren}
%\include{form}
%\include{allgtips}
%
%\bibliographystyle{alpha}  % Schlüssel als Buchstaben
%\bibliography{literaturverzeichnis}      % Literaturverzeichnis

\end{document}


